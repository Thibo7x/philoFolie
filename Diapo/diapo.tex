\documentclass{beamer}

\usepackage{gensymb}
\input{enteteBeamerLatex}
\usepackage{listings}
\usepackage[babel=true]{csquotes}
\lstset{language=Python, tabsize=2, breaklines=true, showstringspaces=false}

\useoutertheme{infolines}
\setbeamersize{text margin left=1cm,text margin right=1cm}

\title{La folie des grandeurs}
\subtitle{Le monde est fou}
\author{Tristan, Alexis, Thomas, Thibaut}

\begin{document}

\begin{frame}
  \titlepage
\end{frame}

\begin{frame}
    \frametitle{Sommaire}
    \begin{multicols}{2}
      {
		\setcounter{tocdepth}{1}
        \tableofcontents
      }
    \end{multicols}
\end{frame}

\section{La folie}

\begin{frame}
  \begin{displayquote}
    Folie~: Dérèglement mental, démence : Sombrer dans la folie.

    Démence~: (latin~: demens, folie) Sérieuse perte ou réduction des capacités cognitives suffisamment importante pour retentir sur la vie d'un individu et entraîner une perte d'autonomie.
  \end{displayquote}
\end{frame}

\begin{frame}
  \begin{itemize}
    \item réaction inappropriée ou disproportionnée face à la perception de la réalité. Relatif à un individu ou un groupe d'individus~;
    \item Action sans but, homme lunatique.
  \end{itemize}
\end{frame}

\section{Le monde est fou}

\begin{frame}{Problématique}
  \begin{center}
    La folie reflète-t-elle une vision de groupe ?
  \end{center}
\end{frame}

\begin{frame}{Société}
  La folie est relative à la société ou à un groupe d'individus.
  Chaque société pense l'autre folle. Jugement relatif.

  Sortir de la norme ou de la majorité.

  Penser comme tout le monde est un risque de mal interpréter la réalité~: actions inappropriées, aliénation à la masse.
\end{frame}

\begin{frame}{Sectes}
  Seul contre le monde~: raisonnement disproportionné, souvent illogique.
\end{frame}

\section{Folie des grandeurs}

\begin{frame}{Problématique}
  \begin{center}
    Les désirs de l'homme sont-ils sans fin ?
  \end{center}
\end{frame}

\begin{frame}{Démesure}
  Démesure~: sortir de la mesure~; indéfini.

  Relatif aux sociétés et à leurs époques, chercher plus grand.
\end{frame}

\begin{frame}{Besoins de l'homme}
  Besoins de l'homme artificiel et absurdité.

  But vain, sans fin.
\end{frame}

\subsection{Stade 1 : Besoins artificiels}
\begin{frame}{Stade 1 : Besoins artificiels}
  \begin{itemize}
    \item l'homme ne se contente jamais de ce qu'il possède~;
    \item but infini, flou, absurde, quête perdue d'avance~;
    \item démence~;
    \item demesure, originalité dans les besoins~;
    \item culture capitaliste et économie de marché.
  \end{itemize}
\end{frame}


\subsection{Stade 2 : Isolement et sectes}
\begin{frame}{Stade 2 : Isolement et sectes}
  \begin{center}
    Le refus du monde s'exprime à travers le refus de la masse : exclusion.
    \includegraphics[width=10cm]{../Images/secte.png}
  \end{center}
\end{frame}

\subsection{Stade 3 : Destruction}
\begin{frame}{Stade 3 : Destruction}

\end{frame}


\section{La société}

\subsection{Une humanité qui se perd}
\begin{frame}{Une humanité qui se perd}
  Plus le temps de vivre, temps impossible à sacrifier.
  Concept d'identité, oppresseur.
\end{frame}

\subsection{Productivisme et consumérisme}
\begin{frame}{Productivisme et consumérisme}
  Sur-production et sur-consommation.
  Alimentation et gaspillage.
  \begin{center}
    \includegraphics[width=10cm]{../Images/urss.png}
  \end{center}
\end{frame}

\subsection{Auto-glorification culturelle}
\begin{frame}{Auto-glorification culturelle}
  Clichés sur les autres : sentiment de supériorité et nationalismes
\end{frame}


\section{Grandeur et démesure}

\subsection{Repousser les frontières au-delà de la terre}
\begin{frame}{Repousser les frontières au-delà de la terre}
  Espace
\end{frame}

\subsection{Pas de génie sans folie : la recherche d'un autre idéal}
\begin{frame}{Pas de génie sans folie : la recherche d'un autre idéal}
  Folie créatrice
\end{frame}

\subsection{Le monde est condamné à être fou}
\begin{frame}{Le monde est condamné à être fou}
  On glorifie ce qui sort de l'ordinaire pcq notre modèle globalisé nous force à nous démarquer : c'est la folie qui nous fait évoluer.

\end{frame}

\end{document}
