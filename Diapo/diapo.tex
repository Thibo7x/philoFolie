\documentclass{beamer}

\usepackage{gensymb}
\input{enteteBeamerLatex}
\usepackage{listings}
\usepackage[babel=true]{csquotes}
\lstset{language=Python, tabsize=2, breaklines=true, showstringspaces=false}

\useoutertheme{infolines}
\setbeamersize{text margin left=1cm,text margin right=1cm}

\title{La folie des grandeurs}
\subtitle{Le monde est fou}
\author{Alexis, Thomas, Thibaut, Tristan}

\begin{document}

\begin{frame}
  \titlepage
\end{frame}

\begin{frame}
    \frametitle{Sommaire}
    \begin{multicols}{2}
      {
		\setcounter{tocdepth}{1}
        \tableofcontents
      }
    \end{multicols}
\end{frame}

\section{Pourquoi le monde serait-il fou ?}

\subsection{La folie}

\begin{frame}
  \begin{displayquote}
    Folie~: Dérèglement mental, démence : Sombrer dans la folie.

    Démence~: (latin~: demens, folie) Sérieuse perte ou réduction des capacités cognitives suffisamment importante pour retentir sur la vie d'un individu et entraîner une perte d'autonomie.
  \end{displayquote}
\end{frame}

\begin{frame}
	\begin{itemize}
		\item réaction inappropriée ou disproportionnée face à la perception de la réalité. Relatif à un individu ou un groupe d'individus~;
		\item action irrationelle~;
		\item action non incluse dans la norme.
	\end{itemize}
\end{frame}


\subsection{La vision de groupe}

\begin{frame}
	Folie~: différent de la norme.
	
	Norme relative à un groupe.
\end{frame}

\begin{frame}
	Norme comme vision de la majorité.
	
	Normes rationnelles ?
\end{frame}

\begin{frame}
	Instrumentalisation de la norme~:
	\begin{itemize}
	 \item états~: guerre froide, URSS~;
	 \item opinions~: veganisme, contre arguments peu rationnelles.
	\end{itemize}
\end{frame}

\subsection{La folie de la norme} % TODO: revoir le titre

\begin{frame}
	Jugement de la folie difficile, presque impossible.

	Ligne de conduite rationellement folles, normes incohèrentes.
\end{frame}

\begin{frame}
	Les normes sont irrationelles~:
	\begin{itemize}
	 \item critères subjectifs~;
	 \item muable~: critères esthétiques~;
	 \item relative~: dependent de la société.
	\end{itemize}
\end{frame}

\begin{frame}
	Les guerre sont irrationelles~:
	\begin{itemize}
	 \item prix de la vie plus faible que le but de la guerre.
	\end{itemize}
\end{frame}


\section{Comment cette folie s'exprime-t-elle ?}

\subsection{L'économie}

\begin{frame}
	Contrôle d'un travail réel par de l'argent virtuel, par exemple les fluctuations de la bourse.

	La variation de la valeur d'un travail est-elle rationelle ?
\end{frame}

\subsection{La démesure}

\begin{frame}
  Démesure~: sortir de la mesure~; indéfini.

  Relatif aux sociétés et à leurs époques, toujours chercher plus grand.

	Si la démesure est observée elle est alors mesurable~: démesure restreinte à l'imagination ?
\end{frame}

\begin{frame}
	Démesure par la grandeur~:
	\begin{itemize}
	 \item besoins de l'homme plus grands et nombreux~;
	 \item productivisme et consumérisme~;
	 \item comportement inapproprié~: réflechir pour sa survie~: inutile, gaspillage.
	\end{itemize}
\end{frame}


\section{Conclusion}

\begin{frame}
	Le monde est fou, et sa folie s’exprime de différentes manières.

	Peut-on guérir le monde de ses maux ?
\end{frame}


\end{document}





























\section{Le monde est fou}

\begin{frame}{Problématique}
  \begin{center}
    La folie reflète-t-elle une vision de groupe ?
  \end{center}
\end{frame}

\begin{frame}{Société}
  La folie est relative à la société ou à un groupe d'individus.
  Chaque société pense l'autre folle. Jugement relatif.

  Sortir de la norme ou de la majorité.

  Penser comme tout le monde est un risque de mal interpréter la réalité~: actions inappropriées, aliénation à la masse.
\end{frame}

\begin{frame}{Sciences alternatives}
	\begin{itemize}
	 \item le scientisme est une vision du monde, apparue au XIXe siècle~;
	 \item Énergie libre
		\begin{center}
			\includegraphics[width=5cm]{../Images/energie_libre.png}
		\end{center}

	\end{itemize}

\end{frame}


\begin{frame}{Sectes}
  Seul contre le monde~: raisonnement disproportionné, souvent illogique.
  \begin{center}
    \includegraphics[width=5cm]{../Images/secte.png}
  \end{center}
\end{frame}

\begin{frame}{Le monde est condamné à être fou}
  On glorifie ce qui sort de l'ordinaire pcq notre modèle globalisé nous force à nous démarquer : c'est la folie qui nous fait évoluer.

\end{frame}

\section{Folie des grandeurs}

\begin{frame}{Problématique}
  \begin{center}
    Les désirs de l'homme sont-ils sans fin ?
  \end{center}
\end{frame}

\begin{frame}{Démesure}
  Démesure~: sortir de la mesure~; indéfini.

  Relatif aux sociétés et à leurs époques, chercher plus grand.
\end{frame}

\begin{frame}{Besoins artificiels}
  \begin{itemize}
    \item l'homme ne se contente jamais de ce qu'il possède~;
    \item but infini, flou, absurde, quête perdue d'avance~;
    \item démence~;
    \item demesure, originalité dans les besoins.
  \end{itemize}
\end{frame}

\begin{frame}{Productivisme et consumérisme}
  Sur-production et sur-consommation.
  Alimentation et gaspillage.
  \begin{center}
    \includegraphics[width=5cm]{../Images/urss.png}
    \includegraphics[width=5cm]{../Images/productivisme.jpeg}
	\end{center}
\end{frame}

\begin{frame}{Menace scientifique : exemple de l'armement}
  \begin{center}
    Nous avons la capacité de détruire plusieurs fois la planète.
    \includegraphics[width=10cm]{../Images/bombe.jpg}
  \end{center}

\end{frame}

\begin{frame}{Conclusion}
  \begin{center}
    Les désirs de l'homme semblent sans fin. \\
    $=>$ changer de paradigme ?

  \end{center}

\end{frame}

\end{document}




\section{La société}

\subsection{Une humanité qui se perd}
\begin{frame}{Une humanité qui se perd}
  Plus le temps de vivre, temps impossible à sacrifier.
  Concept d'identité, oppresseur.
\end{frame}

\subsection{Productivisme et consumérisme}



\section{Grandeur et démesure}

\subsection{Repousser les frontières au-delà de la terre}
\begin{frame}{Repousser les frontières au-delà de la terre}
  Espace
\end{frame}
