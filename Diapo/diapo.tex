\documentclass{beamer}

\usepackage{gensymb}
\input{enteteBeamerLatex}
\usepackage{listings}
\usepackage[babel=true]{csquotes}
\lstset{language=Python, tabsize=2, breaklines=true, showstringspaces=false}

\useoutertheme{infolines}
\setbeamersize{text margin left=1cm,text margin right=1cm}

\title{La folie des grandeurs}
\subtitle{Le monde est fou}
\author{Tristan, Alexis, Thomas, Thibaut}

\begin{document}

\begin{frame}
  \titlepage
\end{frame}

\begin{frame}
    \frametitle{Sommaire}
    \begin{multicols}{2}
      {
		\setcounter{tocdepth}{1}
        \tableofcontents
      }
    \end{multicols}
\end{frame}

\section{La folie}

\subsection{Définition}
\begin{frame}{Définition}

\end{frame}

\subsection{Stade 1 : Besoins artificiels}
\begin{frame}{Stade 1 : Besoins artificiels}

\end{frame}


\subsection{Stade 2 : Isolement et sectes}
\begin{frame}{Stade 2 : Isolement et sectes}

\end{frame}

\subsection{Stade 3 : Destruction}
\begin{frame}{Stade 3 : Destruction}

\end{frame}


\section{La société}

\subsection{Une humanité qui se perd}
\begin{frame}{Une humanité qui se perd}
  Plus le temps de vivre
  Concept d'identité, qui nous oppresse en nous enfermant dans des cases
\end{frame}

\subsection{Productivisme et consumériste}
\begin{frame}{Productivisme et consumériste}

\end{frame}

\subsection{Auto-glorification culturelle}
\begin{frame}{Auto-glorification culturelle}
  Clichés sur les autres : sentiment de supériorité et nationalismes
\end{frame}


\section{Grandeur et démesure}

\subsection{Repousser les frontières au-delà de la terre}
\begin{frame}{Repousser les frontières au-delà de la terre}
  Espace
\end{frame}

\subsection{Pas de génie sans folie : la recherche d'un autre idéal}
\begin{frame}{Pas de génie sans folie : la recherche d'un autre idéal}
  Folie créatrice
\end{frame}

\subsection{Le monde est condamné à être fou}
\begin{frame}{Le monde est condamné à être fou}
  On glorifie ce qui sort de l'ordinaire pcq notre modèle globalisé nous force à nous démarquer
\end{frame}


\end{document}
